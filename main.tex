\documentclass[10pt,aspectratio=169]{beamer}

\usetheme[progressbar=frametitle]{metropolis}

\usepackage{appendixnumberbeamer}
\usepackage{booktabs}
\usepackage[scale=2]{ccicons}

\usepackage[normalem]{ulem}
\usepackage{subcaption}
\usepackage{tikz}
\usepackage{minted}
\usepackage{pgfplots}
\usepgfplotslibrary{dateplot}
\usepackage{csquotes}
\usepackage{amssymb}
\usepackage{pifont}
\usepackage{xcolor}
\newcommand{\cmark}{\ding{51}}
\newcommand{\xmark}{\ding{55}}
\newcommand{\done}{\rlap{$\square$}{\raisebox{2pt}{\large\hspace{1pt}\textcolor{green}{\cmark}}}\hspace{-2.5pt}}
\newcommand{\wontfix}{\rlap{$\square$}{\large\hspace{1pt}\textcolor{red}{\xmark}}}
\newcommand{\partialdone}{\rlap{$\square$}{\raisebox{2pt}{\large\hspace{1pt}\textcolor{orange}{\cmark}}}\hspace{-2.5pt}}

\usepackage[backend=biber,style=numeric, citestyle=ieee]{biblatex}
\addbibresource{references.bib}

\usepackage{pgfpages}
% \setbeamertemplate{note page}{\vspace*{0.75cm}\insertnote}
% \setbeameroption{show notes on second screen=right}
% \setbeameroption{show notes}
\setbeameroption{hide notes}


\title{Assessing the Viability of Rust in HPC}
\subtitle{3rd year project presentation}
\author{Edmund Goodman}
\date{\today}


% ================================ %

\begin{document}

\maketitle
    \note[itemize]{
        \item Explain project is not just assessing Rust in HPC, but also thinking about workflows and pragmatic viability\
        \item Agenda will be:
        \begin{enumerate}
            \item Project background
            \item Translation process
            \item Equivalence checking
            \item Performance analysis and demo
            \item Project management
            \item Conclusion
        \end{enumerate}
    }

% \begin{frame}{Agenda}
%   \setbeamertemplate{section in toc}[sections numbered]
%   \tableofcontents%[hideallsubsections]
% \end{frame}

\begin{frame}{Before we begin...}
\begin{itemize}
    \item The performance analysis section will analyse benchmark runs
    \vspace*{0.5cm}
    \item I'm going to use a tool I built to start some of on DCS batch compute
        % \note<2>{\texttt{ssh kudu}\newline}
        % \note<2>{\texttt{cd ~/Desktop/project/hpccg-rs/hpc-multibench}\newline}
        % \note<2>{\texttt{poetry run python3 -m hpc\_multibench -y yaml\_examples/kudu/strong\_weak\_scaling.yaml record}}
    \begin{itemize}
        \item \alert{We will come back and explore them later, once they have completed!}
    \end{itemize}
\end{itemize}
\end{frame}

\section{Project Background}

\begin{frame}{The Rust Language \ i}
    % \begin{tikzpicture}[remember picture,overlay]
    %     \node[xshift=-1.2cm,yshift=-1.8cm] at (current page.north east) {\includegraphics[width=0.15\textwidth]{images/ferris.png}};
    % \end{tikzpicture}
    \begin{itemize}
        \item Originally developed by Mozilla, and released in 2010
            \note[item]{Relatively young language, 14 years to C++'s 45\newline}
        \item Its website claims \cite{RustProgrammingLanguage} it is:
        \begin{itemize}
            \item Performant
                \note[item]{No garbarge collection needed}
                \note[item]{Offers low-level abstractions for the computer hardware, facilitating performant code\newline}
            \item Productive
                \note[item]{Robust toolchain, including canonical linters (\texttt{clippy}), auto-formatters (\texttt{rustfmt}), and a centralised package repository (\texttt{crates.io})\newline}
            \item Reliable
                \note[item]{Rich type system, with foundation in programming languages like Haskell}
                \note[item]{Guarantees of safety by preventing undefined behaviour at compile time\newline}
        \end{itemize}
        \vspace*{0.25cm}
        \item Eliminates ``undefined behaviour'' by compiler rules
            \note[item]{Specifically, its ownership model which works by counting mutable and immutable references to data}
        \begin{itemize}
            \item Encompasses memory and thread safety
            \item Over 70\% of security bugs in the Chromium project related to memory safety \cite{MemorySafety} % serious security bugs
        \end{itemize}
        \vspace*{0.25cm}
        \item Voted ``most loved'' programming language in StackOverflow's developer survey \cite{StackOverflowDeveloper}
    \end{itemize}
\end{frame}

\begin{frame}{The Rust Language \ ii}
    % \begin{tikzpicture}[remember picture,overlay]
    %     \node[xshift=-1.2cm,yshift=-1.8cm] at (current page.north east) {\includegraphics[width=0.15\textwidth]{images/ferris.png}};
    % \end{tikzpicture}
    Has been in the news even in the last month due to:
    \begin{itemize}
        \item White House Office of the National Cyber Director publishing a press release ``Future Software Should Be Memory Safe'' \cite{PressReleaseFuture2024}  % (February 26th)
            \note[item]{Explicitly mentioning moving critical codebases to Rust as a part of US national cyber security policy}
        \item Google pledging \$1 million to improve Rust/C++ interoperability \cite{ImprovingInteroperabilityRust}  % (February 5th)
        \begin{itemize}
            \item<2-> \alert{Blog post mentions tooling such as \texttt{autocxx} as part of this effort}
                \note[item]{\texttt{autocxx}, is a tool to which I had already pull requested integration tests in the course of the project}
        \end{itemize}
    \end{itemize}

    \vspace{0.5cm}
    \only<3->{
        \begin{figure}[H]
            \includegraphics[width=0.95\textwidth]{images/autocxx_pr.png}
            \caption{A screenshot of my merged pull request into \texttt{autocxx}}
            \label{fig:warwick_mantevo_link}
        \end{figure}
    }
\end{frame}

\begin{frame}{High-Performance Computing}
    \begin{displayquote}
        \vspace{0.2cm}
        ``Technology that uses clusters of powerful processors, working in \textbf{parallel}, to process massive multidimensional datasets (\textbf{big data}) and solve complex problems at \textbf{extremely high speeds}.'' -- IBM \cite{WhatHPCIntroduction}
    \end{displayquote}
    % \vspace*{0.25cm}
    % \begin{itemize}
    %     \item Often referred to by the initialism ``HPC''
    %     \item Key characteristics from definition:
    %     \begin{itemize}
    %         \item Parallel computation
    %             \note[item]{Within UMA machines with vectorisation and threading}
    %             \note[item]{Across NUMA clusters over TCP/IP networks\newline}
    %         \item Large data-sets
    %         \item \alert{High speeds}
    %             \note[item]{Overall goal of HPC is going as fast as possible...}
    %     \end{itemize}
    % \end{itemize}
\end{frame}

\begin{frame}{Mantevo Suite}
    \begin{displayquote}
        \vspace{0.2cm}
        ``[Mini-apps are] \textbf{small} software programs whose \textbf{performance characteristics model full-scale applications}, yet require only \textbf{a fraction of the lines of code}, making [them] easier to study, design, and rewrite.'' -- Mantevo Suite summary \cite{heroux2013mantevo}.
    \end{displayquote}
    \vspace*{0.25cm}
    \begin{itemize}
        \item In the early 2010s, clock speeds had stalled and parallelisation became key
        \item Physical limitations forced hardware/software co-design to meet performance goals
        \item Pioneered the concept of mini-apps (also called proxy applications):
        \begin{itemize}
            % \item Small codebases for ease of use
            % \item Similar performance characteristics to larger applications
            % \item Facilitate co-design
            %     \note[item]{Easy predictors of full application performance, which can be deployed early in hardware development}
            \item Can be re-purposed to assess software instead of hardware
        \end{itemize}
    \end{itemize}
\end{frame}

\begin{frame}{Warwick's Link to the Mantevo Suite}
    \begin{figure}[H]
        \includegraphics[width=0.35\textwidth]{images/warwick_mantevo_link.png}
        \caption{A screenshot of an appendix to the original Mantevo Suite paper, showing Stephen Jarvis' support for the project \cite{heroux2013mantevo}.}
        \label{fig:warwick_mantevo_link}
    \end{figure}
        \note[item]{Perhaps this helps justify mini-apps as an interesting thing to think about?}
\end{frame}

\begin{frame}{High Performance Computing Conjugate Gradients (HPCCG) \ i}
    \begin{columns}[onlytextwidth]
        \column{0.5\textwidth}
            \begin{itemize}
                \item ``The original Mantevo mini-app'' \cite{MantevoHPCCG2023}
                \vspace{0.5cm}
                \item Designed to be ``the best approximation to an unstructured implicit finite element or finite volume application in 800 lines or fewer.'' \cite{heroux2013mantevo}
                \item Based on the iterative method of gradient descent for conjugate gradients first proposed by Hestnes and Steifel in 1952 \cite{hestenesMethodsConjugateGradients1952}
                \item A small C++ codebase with dependent only on \texttt{OpenMP} and \texttt{MPI}
            \end{itemize}
        \column{0.5\textwidth}
            \begin{figure}[H]
                \includegraphics[width=0.75\textwidth]{images/acacgs_silo_output.png}
                \captionsetup{width=.9\linewidth}
                \caption{A screenshot of a silo visualisation from ACACGS, a C translation of HPCCG by Richard Kirk used in the CS257 coursework.}
                \label{fig:warwick_mantevo_link}
            \end{figure}
    \end{columns}
\end{frame}

\begin{frame}{High Performance Computing Conjugate Gradients (HPCCG) \ ii}
    % TODO: Regenerate images with terminal background set to #fafafa
    \begin{figure}
        \begin{subfigure}[c]{.45\textwidth}\centering
            \includegraphics[width=0.95\textwidth]{images/hpccg-run.png}
            \label{fig:hpccg-run}
        \end{subfigure}%
        \begin{subfigure}[c]{.45\textwidth}\centering
            \includegraphics[width=0.95\textwidth]{images/hpccg-perf.png}
            \label{fig:hpccg-perf}
        \end{subfigure}
        \caption{The output from running HPCCG over a small mesh}
    \end{figure}
        \note[item]{HPCCG compiles to a standalone binary}
        \note[item]{Can be optionally compiled with OpenMP and MPI}
        \note[item]{This binary takes three parameters, the x y and z sizes of the mesh it runs on}
        \note[item]{It prints out its state as it performs iterations}
        \note[item]{When its done, it gives a performance summary}
\end{frame}

\begin{frame}{P$^3$HPC - not just performance!}
    % TODO: Re-work this slide? Add some images or something?
    % https://tex.stackexchange.com/questions/10060/how-to-draw-kiviat-diagrams for C++/Rust
    % Or an iron triangle breakdown

    % \begin{itemize}
    %     % \item <2-> Performance
    %     \item \alert<2>{Performance}
    %         \note[item]{This is something the project assesses!}
    %         \note[item]{Rust needs to nearly or as fast as C++ to be viable\newline}
    %     \vspace*{1cm}
    %     % \item <3-> Productivity
    %     \item \alert<3>{Productivity}
    %         \note[item]{Ergonomic syntax, most loved language with rich type system etc.}
    %         \note[item]{Good, canonical toolchain (\texttt{cargo}, \texttt{rustfmt}, \texttt{clippy}, ...)}
    %         \note[item]{Compiler check which make writing bug-free code easier\newline}
    %     \vspace*{1cm}
    %     % \item <4-> Portability
    %     \item \alert<4>{Portability}
    %         \note[item]{Rust toolchain is very portable (\texttt{rustc} / \texttt{LLVM} vs \texttt{gcc}, \texttt{icpc}, \texttt{clang}, ...)} % not like c++ with 5+ different proprietary compilers with slightly different flags
    %         \note[item]{Still capable of machine-specific optimisations with native architecture flag}
    % \end{itemize}
    
    \begin{figure}[H]
        \includegraphics[width=0.5\textwidth]{images/excalidraw_p3_triangle.png}
        \vspace*{0.75cm}
        \label{fig:warwick_mantevo_link}
        % \caption{A diagram of three important aspects of high-performance software.}
    \end{figure}

    
    % \textbf{Why might Rust be a good fit for HPC?}
\end{frame}


% \begin{frame}{Existing work}
%     \begin{itemize}
%         \item Previous publications tend to address only performance
%         \item HPC is not only about performance
%         \item \alert{Aim to provide a holistic view of Rust's viability for HPC}
%     \end{itemize}
% \end{frame}

\section{Translation Process}

\begin{frame}{C++ vs Rust}
    \begin{columns}[onlytextwidth]
        \centering
        \column{0.5\textwidth}
            \begin{itemize}
                \item There are more valid C++ programs\newline than (safe) Rust programs!
                    \note[item]{In a similar sense to moving from un-typed to type lambda calculi}
                    \note[item]{Could be considered an extension to Barendregt’s Lambda cube}
                \item Rust explicitly prohibits undefined behaviour by default
                \begin{itemize}
                    \item This is normally a good thing!
                \end{itemize}
                \item Translation is less trivial
                \begin{itemize}
                    \item Direct mapping might not exist
                    \item Can resort to \mintinline{rust}{unsafe} % TODO check other texttt tags
                \end{itemize}
            \end{itemize}
        \column{0.5\textwidth}
            \begin{figure}[H]
                \includegraphics[width=0.65\textwidth]{images/excalidraw_programs_venn.png}
                % \captionsetup{width=.9\linewidth}
                \caption{A diagram to show the relative powers of increasingly restrictive program models.}
            \label{fig:warwick_mantevo_link}
        \end{figure}
    \end{columns}
    % The program model is different
    % The domain of valid C++ programs is a strict superset of Rust programs
    % This is helpful, as it means you can't write programs with undefined behaviour
    % This makes translation more difficult
\end{frame}

% \begin{frame}{Writing Performant Rust}
%     % TODO: Drop this slide
%     % There are a number of tricks you can do to make Rust more performant
%     % Many are enumerated in the Rust Performance Book
% \end{frame}

\begin{frame}[fragile]{Fearless Parallelism in Rust}
    \begin{itemize}
        \item<1-> One of Rust's key selling points is ``Fearless Parallelism''
        \begin{itemize}
            \item Many concurrency bugs moved from runtime to compile-time
                \note[item]{Ownership and type systems provide strong guarantees against undefined behaviour}
                \note[item]{Non thread-safe code often doesn't even compile, reducing the need to debug flaky race conditions}
            \item Ubiquitous crates provide very high-level abstractions
        \end{itemize}
    \end{itemize}    
    \vspace*{0.15cm}

    \begin{figure}
        \begin{subfigure}[c]{.55\textwidth}\centering
            \begin{minted}[escapeinside=||,linenos]{c++}
double dot_product (
    int n, double* x, double* y
) {
    double result = 0.0;
    for (int i = 0; i < n; i++) {
        result += x[i] * y[i];
    }
    return result;
}
            \end{minted}
            \label{fig:cpp-ddot-serial}
            \caption{C++ parallel implementation}
        \end{subfigure}%
        \begin{subfigure}[c]{.45\textwidth}\centering
            \begin{minted}[escapeinside=||,linenos]{rust}
pub fn dot_product(
lhs: &[f64], rhs: &[f64]
) -> f64 {
lhs.iter()
    .zip(rhs.iter())
    .map(|(x, y)| x * y)
    .sum()
}
            \end{minted}
            \label{fig:rust-ddot-serial}
            \vspace*{0.5cm}
            \caption{Rust serial implementation}
        \end{subfigure}
    \end{figure}
\end{frame}

\begin{frame}[fragile, noframenumbering]{Fearless Parallelism in Rust}
    \begin{itemize}
        \item<1-> One of Rust's key selling points is ``Fearless Parallelism''
        \begin{itemize}
            \item Many concurrency bugs moved from runtime to compile-time
                \note[item]{Ownership and type systems provide strong guarantees against undefined behaviour}
                \note[item]{Non thread-safe code often doesn't even compile, reducing the need to debug flaky race conditions}
            \item Ubiquitous crates provide very high-level abstractions
        \end{itemize}
    \end{itemize}    
    \vspace*{0.15cm}

    \begin{figure}
        \begin{subfigure}[c]{.55\textwidth}\centering
            \begin{minted}[escapeinside=||,linenos]{c++}
double dot_product (
    int n, double* x, double* y
) {
    double result = 0.0;
    #pragma omp parallel for
    for (int i = 0; i < n; i++) {
        result += x[i] * y[i];
    }
    return result;
}
            \end{minted}
            \label{fig:cpp-ddot-openmp-race}
            \caption{C++ parallel implementation}
            \only<2> {
                \begin{tikzpicture}[remember picture,overlay]
                    \draw[red, very thick] (-2.4,3.1) rectangle node[below, xshift=3cm, yshift=-0.3cm] {Data race!} (-0.4,2.5);
                \end{tikzpicture}
            }
        \end{subfigure}%
        \begin{subfigure}[c]{.45\textwidth}\centering
            \begin{minted}[escapeinside=||,linenos]{rust}
pub fn dot_product(
lhs: &[f64], rhs: &[f64]
) -> f64 {
lhs.iter()
    .zip(rhs.iter())
    .map(|(x, y)| x * y)
    .sum()
}
            \end{minted}
            \label{fig:rust-ddot-serial-2}
            \vspace*{0.5cm}
            \caption{Rust serial implementation}
        \end{subfigure}
    \end{figure}
\end{frame}

\begin{frame}[fragile, noframenumbering]{Fearless Parallelism in Rust}
    \begin{itemize}
        \item<1-> One of Rust's key selling points is ``Fearless Parallelism''
        \begin{itemize}
            \item Many concurrency bugs moved from runtime to compile-time
                \note[item]{Ownership and type systems provide strong guarantees against undefined behaviour}
                \note[item]{Non thread-safe code often doesn't even compile, reducing the need to debug flaky race conditions}
            \item Ubiquitous crates provide very high-level abstractions
        \end{itemize}
    \end{itemize}    
    \vspace*{0.15cm}

    \begin{figure}
        \begin{subfigure}[c]{.55\textwidth}\centering
            \begin{minted}[escapeinside=||,linenos,breaklines]{c++}
double dot_product (
    int n, double* x, double* y
) {
    double result = 0.0;
#pragma omp parallel for reduction(+:result)
    for (int i = 0; i < n; i++) {
        result += x[i] * y[i];
    }
    return result;
}
            \end{minted}
            \label{fig:cpp-ddot-openmp-reduction}
            \vspace*{-0.5cm}
            \caption{C++ fixed parallel implementation}
        \end{subfigure}%
        \begin{subfigure}[c]{.45\textwidth}\centering
            \begin{minted}[escapeinside=||,linenos]{rust}
pub fn dot_product(
lhs: &[f64], rhs: &[f64]
) -> f64 {
lhs.iter()
    .zip(rhs.iter())
    .map(|(x, y)| x * y)
    .sum()
}
            \end{minted}
            \label{fig:rust-ddot-serial-3}
            \vspace*{1.15cm}
            \caption{Rust serial implementation}
        \end{subfigure}
    \end{figure}
\end{frame}

\begin{frame}[fragile, noframenumbering]{Fearless Parallelism in Rust}
    \begin{itemize}
        \item<1-> One of Rust's key selling points is ``Fearless Parallelism''
        \begin{itemize}
            \item Many concurrency bugs moved from runtime to compile-time
                \note[item]{Ownership and type systems provide strong guarantees against undefined behaviour}
                \note[item]{Non thread-safe code often doesn't even compile, reducing the need to debug flaky race conditions}
            \item Ubiquitous crates provide very high-level abstractions
        \end{itemize}
    \end{itemize}    
    \vspace*{0.15cm}

    \begin{figure}
        \begin{subfigure}[c]{.55\textwidth}\centering
            \begin{minted}[escapeinside=||,linenos,breaklines]{c++}
double dot_product (
    int n, double* x, double* y
) {
    double result = 0.0;
#pragma omp parallel for reduction(+:result)
    for (int i = 0; i < n; i++) {
        result += x[i] * y[i];
    }
    return result;
}
            \end{minted}
            \label{fig:cpp-ddot-openmp-reduction-2}
            \vspace*{-0.5cm}
            \caption{C++ fixed parallel implementation}
        \end{subfigure}%
        \begin{subfigure}[c]{.45\textwidth}\centering
            \begin{minted}[escapeinside=||,linenos]{rust}
use rayon::prelude::*; 

pub fn dot_product(
lhs: &[f64], rhs: &[f64]
) -> f64 {
lhs.par_iter()
    .zip(rhs.par_iter())
    .map(|(x, y)| x * y)
    .sum()
}
            \end{minted}
            \label{fig:rust-ddot-rayon}
            \caption{Rust Rayon implementation}
            \begin{tikzpicture}[remember picture,overlay]
                \draw[red, very thick] (-1,3.1) rectangle (1.5,2.5);
                \draw[red, very thick] (-2.5,3.75) rectangle (0,3.15);
            \end{tikzpicture}
        \end{subfigure}
    \end{figure}
\end{frame}



\begin{frame}{Clustered Computing in Rust}
    \begin{itemize}
        \item The Message Passing Interface (\texttt{MPI}) is a common way to distribute work across machines in a HPC computer cluster
        \item \texttt{MPI} has native C++ and FORTRAN implementations of its standard specification
        \item The \texttt{mpi} crate provides bindings from Rust into the C++ implementation
        \begin{itemize}
            \item These bindings are designed to be ``rustic''
            \item They (so far\footnote{I am looking at pull requesting simple examples based on the integration tests}) have quite poor documentation
        \end{itemize}
        % MPI spec
        % TODO: Could add code example?
    \end{itemize}
\end{frame}

% \begin{frame}{Translation Workflow}
%     % TODO: Is this slide needed?
%     % Diagram of loop from translate -> equivalence check -> characterise performance
%     % Ease of these three steps constitute criteria for viability of moving from C++ to Rust in HPC
% \end{frame}




\section{Equivalence Checking}

\begin{frame}{Equivalence Checking Techniques}
    % TODO: Re-work this slide
    \begin{itemize}
        \item Equivalence checking is critical to keep performance comparisons fair
        \item Four different techniques considered
        \begin{enumerate}
            \item<1-> \alert<5>{End-to-end testing}
            \begin{itemize}
                \item Run the program and examine its output
                \item Fairly effective for chaotic systems (like HPCCG), but can miss edge-cases in complex logic % Elucidated weird bug in both versions given non-associativity of floating point operations
            \end{itemize}
            \item<2-> Formal methods
            \begin{itemize}
                \item Techniques such as Hoare Logic \cite{hoareAxiomaticBasisComputer1969} allow formal reasoning about programs
            \end{itemize}
            \item<3-> LLVM analysis
            \begin{itemize}
                \item Since the translated code does the same thing, generated LLVM IR may look similar
            \end{itemize}
            \item<4-> \alert<5>{Unit testing}
            \begin{itemize}
                \item Drive small code units and examine their outputs
                \item Can provide confidence about these units \textit{individually}
                \item Requires manual duplication across both languages, if no tooling exists...
            \end{itemize}
        \end{enumerate}
    \end{itemize}
    % End-to-end testing (is fairly effective for chaotic systems like HPCCG)
    %  - Elucidated weird bug in both versions given non-associativity of floating point operations
    % Unit testing
    % Formal methods (refer to Hoare logic with neat example? conclusion as not too helpful)
    % LLVM analysis (just compare some clang decompiled code? conclusion as could be helpful, but high effort to build such a tool)
\end{frame}

\begin{frame}{Unit Tests with Test-Driven Development}
    \begin{figure}[H]
        \hspace*{-1cm}
        \includegraphics[width=0.45\textwidth]{images/excalidraw_tdd.png}
        \caption{A diagram showing the top-level workflow of the Test-Driven Development methodology.}
        \label{fig:tdd_workflow}
    \end{figure}
    \only<2>{\centering\large\alert{What if we could leverage this during translation...?}}
\end{frame}

\begin{frame}{Novel tool \#1: Polyglotest \ i}
    % TODO: Re-work this slide
    \begin{overprint}
        \onslide<1-3>
            \begin{alertblock}{Key Insight}
                \vspace*{0.25cm}
                Running existing unit tests from the original implementation on the translated one gives strong guarantees of equivalence.
            \end{alertblock}
            \vspace*{1cm}
            \begin{itemize}
                \item Write/use existing unit tests for the original C++ code to drive translated Rust code
                \item How do we do this?
                \begin{itemize}
                    \item<2-> Use a ``foreign function interface'' bridge like \texttt{cxx}
                \end{itemize}
                \item<3-> \textit{But our software output is in Rust, shouldn't we be getting rid of C++ unit tests?}
            \end{itemize}
        \onslide<4->
            \begin{alertblock}{Key Insight}
                \vspace*{0.25cm}
                Running unit tests from the \sout{original} \textit{translated} implementation on the \sout{translated} \textit{original} one gives strong guarantees of equivalence.
            \end{alertblock}
            \vspace*{1cm}
            \begin{itemize}
                \item Write unit tests for the \textit{translated Rust} code to drive \textit{original C++} code
                \begin{itemize}
                    \item Again, use a ``foreign function interface'' bridge like \texttt{autocxx}
                \end{itemize}
                \vspace*{0.5cm}
                \item<5-> \alert{We can now do \textbf{pure} Test-Driven Development to translate our Rust code!}
            \end{itemize}
    \end{overprint}

\end{frame}

\begin{frame}[fragile]{Novel tool \#1: Polyglotest \ ii}
        \begin{columns}[T,onlytextwidth]
            \centering
            \column{0.5\textwidth}
            \begin{minted}[escapeinside=||,linenos,lastline=15]{rust}
include_cpp! {
    #include "ddot.hpp"
    safety!(unsafe_ffi)
    generate!("ddot")
}

#[test]
fn test_ddot() {
    let width = 3;
    let lhs = vec![1.0, 2.0, 3.0];
    let rhs = vec![3.0, 2.0, 1.0];

    let result = if TEST_RUST {
        ddot(width, &lhs, &rhs)
    } else {
        let mut result = 0.0;
        let mut tmp = 0.0;
        unsafe {
            ffi_ddot(
                c_int(width as i32),
                lhs.as_ptr(),
                rhs.as_ptr(),
                &mut result,
                Pin::new(&mut tmp),
            );
        }
        result
    };
    assert_eq!(result, 10.0);
}
            \end{minted}
            \column{0.5\textwidth}
            \begin{minted}[escapeinside=||,linenos,firstline=16]{rust}
include_cpp! {
    #include "ddot.hpp"
    safety!(unsafe_ffi)
    generate!("ddot")
}

#[test]
fn test_ddot() {
    let width = 3;
    let lhs = vec![1.0, 2.0, 3.0];
    let rhs = vec![3.0, 2.0, 1.0];

    let result = if TEST_RUST {
        ddot(width, &lhs, &rhs)
    } else {
        let mut result = 0.0;
        let mut tmp = 0.0;
        unsafe {
            ffi_ddot(
                c_int(width as i32),
                lhs.as_ptr(),
                rhs.as_ptr(),
                &mut result,
                Pin::new(&mut tmp),
            );
        }
        result
    };
    assert_eq!(result, 10.0);
}
            \end{minted}
        \end{columns}
\end{frame}




\section{Performance Analysis}
\begin{frame}{Performance Analysis Techniques}
    % TODO: Re-work this slide
    \begin{itemize}
        \item Near-peer performance is key for Rust to be viable
        \item Two different ways of assessing performance
        \begin{enumerate}
            \item Performance profilers
            \item Direct comparison
        \end{enumerate}
    \end{itemize}
\end{frame}

\begin{frame}{Analysis with \texttt{perf} \ i}
    \begin{figure}[H]
        \includegraphics[width=0.85\textwidth]{images/perf_annot_cmp.png}
        \caption{A screenshot of the \texttt{perf} tool showing hotspots in the C++ (left) and Rust (right) sparse matrix-vector multiplication kernels.}
        \label{fig:perfAnnotCmp}
    \end{figure}
\end{frame}

\begin{frame}{Analysis with \texttt{perf} \ ii}
    \begin{figure}[H]
        \includegraphics[width=0.65\textwidth]{images/perf_annot_labelled.png}
        \caption{An annotated screenshot of the \texttt{perf} tool showing how ownership checks dominate the proportion of time taken in the Rust implementation, reducing total performance.}
        \label{fig:perfAnnotLabelled}
    \end{figure}
\end{frame}

% \begin{frame}{Analysis with Intel\textsuperscript{\textregistered}\ vTune\texttrademark}
%     % TODO: Drop this slide
%     \begin{figure}[H]
%         \includegraphics[width=0.85\textwidth]{images/intelvtune.png}
%         \caption{A screenshot of the summary page of Intel\textsuperscript{\textregistered}\ vTune\texttrademark\ when run on the Rust implementation.}
%         \label{fig:intelvtune}
%     \end{figure}
% \end{frame}

\begin{frame}{Novel tool \#2: HPC MultiBench}
    \begin{itemize}
        \item<1-> Manually spawning many similar jobs with different configurations is tedious!
        \begin{itemize}
            \item Time consuming
            \item Susceptible to human error
            \item Discourages duplicating results for statistical confidence
        \end{itemize}
        \vspace{0.5cm}
        \item<2-> HPC MultiBench is a Python tool to define, run, and analyse batch compute jobs via Slurm from a convenient YAML format
    \end{itemize}
\end{frame}

\begin{frame}{Returning to the live demo}
    By the end of this demo, I will have shown:
    \begin{enumerate}
        \item <1-> How a test plan can be defined using a YAML file
        \item <2-> How runs to exercise a metric over a set of configurations can automatically be created
        \begin{itemize}
            \item Remember the beginning of the talk...?
        \end{itemize}
        \item <3-> How metrics from completed runs can be aggregated and visualised
        % TODO: Get documentation working!
        % \item <4-> Documentation for the tool
    \end{enumerate}
    \vspace*{0.5cm}
    \alert{It will also show all the directly measured results for comparing C++ and Rust performance}
\end{frame}



% Serial vs Threading
% Threading vs mpi vs hybrid

% serial
% threading
% mpi
% hybrid
 % CPP / Kokkos / Rust

% multinodal (live demo)
% strong/weak scaling (live demo)

% Summary slide to explicitly answer the question


% \begin{frame}{Overview of results \ i}
%     % Strong/weak scaling
% \end{frame}

% \begin{frame}{Overview of results \ ii}
%     % CPP / Kokkos / Rust
% \end{frame}

% \begin{frame}{Parallelism strategies \ iii}
%     % All the different versions against each other
% \end{frame}


% \begin{frame}{Roofline Models}
% \end{frame}

\begin{frame}{Applying Roofline Models to HPCCG}
    % Add `lscpu` for my machine alongside it
    \begin{figure}[H]
        \includegraphics[width=0.65\textwidth]{images/Athena_ERT_generated_roofline.pdf}
        \caption{A Roofline Diagram generated using the Empirical Roofline Toolkit \cite{yangEmpiricalRooflineMethodology2018}, with data-points measured using LIKWID \cite{treibigLIKWIDLightweightPerformance2012} showing serial and parallel implementations of HPCCG in Rust and C++.}
        \label{fig:intelvtune}
    \end{figure}
\end{frame}

\begin{frame}{Summary of Performance Analysis}
    % TODO: Add summary plot
    \begin{itemize}
        \item Rust is non-negligbly slower than C++
        \item In certain applications, this may be acceptable due to other commensurate benefits
    \end{itemize}
\end{frame}



\section{Project management}
% Can be a little light, but very very required

\begin{frame}{Agile Workflows}
    \begin{figure}[h]
        \centering
        \includegraphics[width=0.65\textwidth]{images/excalidraw_agile.png}
        \caption{A diagram of a typical workflow for the Scrum flavour of Agile development.}
        \label{fig:specification_gantt_chart}
    \end{figure}

    \only<2->{
        \begin{tikzpicture}[remember picture,overlay]
            \node[xshift=-2.5cm,yshift=-1.75cm] at (current page.north east) {\includegraphics[width=0.25\textwidth]{images/github_commit_plot.png}};
        \end{tikzpicture}
        \begin{tikzpicture}[remember picture,overlay]
            \node[xshift=-1cm,yshift=-2.65cm] at (current page.north east) {\includegraphics[width=0.115\textwidth]{images/github_streak.png}};
        \end{tikzpicture}
        \vspace*{-0.75cm}
    }

    \begin{itemize}
        \item Weekly supervisor meetings, which bookend sprints
            \note[item]{Act as both sprint retrospective and planning}
            \note[item]{Agenda written up beforehand, then minutes submitted to Tabula afterwards\newline}
        \item Agile has empowered the project to be goal orientated and flexible to change \cite{beckManifestoAgileSoftware2001}
            \note[item]{Goal orientation has allowed me to make a number of polished software products}
            \note[item]{Flexibility to change as allowed be to adapt the initial plan based on new information and ongoing supervisor discussion\newline}
        \item Daily work tracked in \texttt{git}
            \note[item]{By daily - I really mean daily! Continuous streak of $\geq 1$ commit every day this year}
    \end{itemize}
\end{frame}

\begin{frame}{Specification Timeline}
    \begin{figure}[h]
        \centering
        \includegraphics[width=\textwidth]{images/specification_gantt_chart.png}
        \caption{The original timeline from my specification}
        \label{fig:specification_gantt_chart}
    \end{figure}
\end{frame}

\begin{frame}{Actual Timeline}
    % Consider merging with previous slide
    \begin{figure}[h]
        \centering
        \includegraphics[width=\textwidth]{images/actual_gantt_chart.png}
        \caption{The actual timeline of the project. The progress report was more all-encompassing, parallelisation was easier, and clustering was harder than expected}
        \label{fig:actual_gantt_chart}
    \end{figure}
\end{frame}



\section{Conclusion}

\begin{frame}{Requirements \ i}
    \begin{enumerate}
        \item[\done\ \ 1.]
          Select a target mini-app from ECP proxy applications or UK-MAC
          (\textbf{Must have})
        \item[\done\ \ 2.]
          Fuzz test\footnote{an automated testing technique which uses boundary and erroneous test data as inputs, whilst monitoring for resultant undesired behaviour, originally proposed by Miller \textit{et al.} in 1990 \cite{millerEmpiricalStudyReliability1990}\cite{liangFuzzingStateArt2018}} the possible mini-apps for memory safety issues using static analysis tooling \cite{stepanovMemorySanitizerFastDetector2015}
          (\textbf{Should have}, \textit{depends on 1})
        \item[\done\ \ 3.]
          Build tooling for running Rust unit tests on C++ code
          (\textbf{Could have})
        \item[\done\ \ 4.]
          Write unit tests for the original C++ version of the
          mini-app
          (\textbf{Should have}, \textit{depends on 1, (3)})
        \item[\done\ \ 5.]
          Write direct translation of serial version mini-app from C++ to Rust
          (\textbf{Must have}, \textit{depends on 1})
        \item[\done\ \ 6.]
          Modify serial version of translated code to be idiomatic Rust \cite{endlerMreIdiomaticrust2023} 
          (\textbf{Should have}, \textit{depends on 5})
    \end{enumerate}
\end{frame}

\begin{frame}{Requirements \ ii}
    \begin{enumerate}
        \item[\done\ \ 7.]
          Equivalence check serial translated code by comparing results of end-to-end tests with original C++ code
          (\textbf{Must have}, \textit{depends on 5}))
        \item[\done\ \ 8.]
          Equivalence check serial translated code by applying passing C++ unit tests to rust code
          (\textbf{Must have}, \textit{depends on 4, 5})
        \item[\wontfix\ \ 9.]
          Equivalence check serial translated code with limited formal verification techniques
          (\textbf{Could have}, \textit{depends on 5})
        \item[\partialdone\ 10.]
          Equivalence check serial translated code by comparing generated LLVM IR of the C++ and translated Rust versions
          (\textbf{Could have}, \textit{depends on 5})
        \item[\done\ 11.]
          Modify the serial translated code to allow parallel execution
          (\textbf{Must have}, \textit{depends on 5})
        \item[\done\ 12.]
          Equivalence check parallel translated code via all previous techniques
          (\textbf{Must have}, \textit{depends on 7, 8, (9), (10), 11})
        \item[\done\ 13.]
          Carry out a performance analysis of the serial translated Rust code with the original C++ code
          (\textbf{Must have}, \textit{depends on 5})
    \end{enumerate}
\end{frame}

\begin{frame}{Requirements \ iii}
    \begin{enumerate}
        \item[\done\ 14.]
          Carry out a performance analysis of the parallel translated Rust code with the original C++ code
          (\textbf{Must have}, \textit{depends on 11})
        \item[\done\ 15.]
          Modify the parallel translated code to allow execution across clustered compute resources
          (\textbf{Could have}, \textit{depends on 11})
        \item[\done\ 16.]
          Equivalence check clustered translated code via all previous techniques
          (\textbf{Could have}, \textit{depends on 7, 8, (9), (10), 15})
        \item[\done\ 17.]
          Carry out a performance analysis of the clustered translated Rust code with the original C++ code
          (\textbf{Could have}, \textit{depends on 15})
    \end{enumerate}
\end{frame}

\begin{frame}{Evaluation}
    \begin{itemize}
        \item Achieved all \textbf{M}ust/\textbf{S}hould have specification points, including stretch goal of assessing clustered compute
        \item Successfully ``Assessed the Viability of Rust in HPC'', answering the project question
        \vspace{0.5cm}
        \item<2-> \alert{Actively improved} the viability of Rust in HPC, by:
        \begin{itemize}
            \item Developing tools and workflows to assist in translation efforts
            \item Making open source contributions of tests and documentation
        \end{itemize}
    \end{itemize}
\end{frame}


\begin{frame}{Open Source Work}
    \begin{itemize}
        \item Pull requests to existing projects
        \begin{itemize}
            \item Integration tests for array operations to \texttt{autocxx}
            \item Plan to pull request documentation to both \texttt{autocxx} and \texttt{rs-mpi}
            % kokkos mini-apps add HPCCG Kokkos version
        \end{itemize}
        \item Repositories to release once project has been assessed
        \begin{itemize}
            \item \texttt{hpccg-rs} A Rust translation of the HPCCG mini-application
            \item \texttt{hpccg-kokkos} A Kokkos translation of the HPCCG mini-application
            \item \texttt{polyglotest} A mini-framework for pure test-driven development of Rust code translated from C++
            \item \texttt{HPC\_MultiBench} A tool to spawn and analyse Slurm jobs for performance benchmarking
        \end{itemize}
    \end{itemize}
\end{frame}

\begin{frame}{Future work}
\begin{itemize}
    \item<1-> Further research points if the project were longer:
    \begin{itemize}
        \item Translate a second, larger, mini-app such as MiniMD \cite{osti_1231191} or HPCG \cite{dongarra2015hpcg}
        \item Extend comparisons to include Rust GPU support
        % Better kokkos comparison
    \end{itemize}
    \vspace{1cm}
    \item<2-> \alert{Publish and maintain novel tools developed} as open source repositories
    \begin{itemize}
        \item<3-> If any of them start getting interest, will respond to pull requests/issues as far as possible
    \end{itemize}
    \item<4-> \alert{Aim to submit a paper} based on the project to P3HPC
\end{itemize}
\end{frame}




\appendix

% \begin{frame}[allowframebreaks]{References}
%   % \nocite{*}
%   \bibliographystyle{IEEEtran}
%   \bibliography{references}
% \end{frame}

\begin{frame}[allowframebreaks]{Bibliography}
    \printbibliography[heading=none]
\end{frame}

\section{Backup Slides}

\begin{frame}{Backup Slides \ i}
\end{frame}

% \begin{frame}[fragile]{Typography}
%       \begin{verbatim}The theme provides sensible defaults to
% \emph{emphasize} text, \alert{accent} parts
% or show \textbf{bold} results.\end{verbatim}

%   \begin{center}becomes\end{center}

%   The theme provides sensible defaults to \emph{emphasize} text,
%   \alert{accent} parts or show \textbf{bold} results.
% \end{frame}

% \begin{frame}{Font feature test}
%   \begin{itemize}
%     \item Regular
%     \item \textit{Italic}
%     \item \textsc{SmallCaps}
%     \item \textbf{Bold}
%     \item \textbf{\textit{Bold Italic}}
%     \item \textbf{\textsc{Bold SmallCaps}}
%     \item \texttt{Monospace}
%     \item \texttt{\textit{Monospace Italic}}
%     \item \texttt{\textbf{Monospace Bold}}
%     \item \texttt{\textbf{\textit{Monospace Bold Italic}}}
%   \end{itemize}
% \end{frame}

% \begin{frame}{Multiple images on one slide}
% \begin{figure}
%     \begin{overprint}
%     \onslide<1>\includegraphics{./figure1.png}
%     \onslide<2>\includegraphics{./figure2.png}
%     \onslide<3>\includegraphics{./figure3.png}
%     \onslide<4->\includegraphics{./figure4.png}
%     \end{overprint}
% \end{figure}
% \end{frame}

% \begin{frame}{Columns}
%   \begin{columns}[T,onlytextwidth]
%     \column{0.5\textwidth}
%       Items
%       \begin{itemize}
%         \item Milk \item Eggs \item Potatos
%       \end{itemize}

%     \column{0.5\textwidth}
%       Enumerations
%       \begin{enumerate}
%         \item First, \item Second and \item Last.
%       \end{enumerate}
%   \end{columns}
% \end{frame}

% \begin{frame}{Animation}
%   \begin{itemize}[<+- | alert@+>]
%     \item \alert<4>{This is\only<4>{ really} important}
%     \item Now this
%     \item And now this
%   \end{itemize}
% \end{frame}

% \begin{frame}{Figures}
%   \begin{figure}
%     \newcounter{density}
%     \setcounter{density}{20}
%     \begin{tikzpicture}
%       \def\couleur{alerted text.fg}
%       \path[coordinate] (0,0)  coordinate(A)
%                   ++( 90:5cm) coordinate(B)
%                   ++(0:5cm) coordinate(C)
%                   ++(-90:5cm) coordinate(D);
%       \draw[fill=\couleur!\thedensity] (A) -- (B) -- (C) --(D) -- cycle;
%       \foreach \x in {1,...,40}{%
%           \pgfmathsetcounter{density}{\thedensity+20}
%           \setcounter{density}{\thedensity}
%           \path[coordinate] coordinate(X) at (A){};
%           \path[coordinate] (A) -- (B) coordinate[pos=.10](A)
%                               -- (C) coordinate[pos=.10](B)
%                               -- (D) coordinate[pos=.10](C)
%                               -- (X) coordinate[pos=.10](D);
%           \draw[fill=\couleur!\thedensity] (A)--(B)--(C)-- (D) -- cycle;
%       }
%     \end{tikzpicture}
%     \caption{Rotated square from
%     \href{http://www.texample.net/tikz/examples/rotated-polygons/}{texample.net}.}
%   \end{figure}
% \end{frame}

% \begin{frame}{Tables}
%   \begin{table}
%     \caption{Largest cities in the world (source: Wikipedia)}
%     \begin{tabular}{lr}
%       \toprule
%       City & Population\\
%       \midrule
%       Mexico City & 20,116,842\\
%       Shanghai & 19,210,000\\
%       Peking & 15,796,450\\
%       Istanbul & 14,160,467\\
%       \bottomrule
%     \end{tabular}
%   \end{table}
% \end{frame}

% \begin{frame}{Blocks}
%   Three different block environments are pre-defined and may be styled with an optional background color.
  
%   \metroset{block=fill}
%   \begin{block}{Default}
%     Block content.
%   \end{block}
%   \begin{alertblock}{Alert}
%     Block content.
%   \end{alertblock}
%   \begin{exampleblock}{Example}
%     Block content.
%   \end{exampleblock}
% \end{frame}

% \begin{frame}{Math}
%   \begin{equation*}
%     e = \lim_{n\to \infty} \left(1 + \frac{1}{n}\right)^n
%   \end{equation*}
% \end{frame}

% \begin{frame}{Line plots}
%   \begin{figure}
%     \begin{tikzpicture}
%       \begin{axis}[
%         mlineplot,
%         width=0.9\textwidth,
%         height=6cm,
%       ]

%         \addplot {sin(deg(x))};
%         \addplot+[samples=100] {sin(deg(2*x))};

%       \end{axis}
%     \end{tikzpicture}
%   \end{figure}
% \end{frame}

% \begin{frame}{Bar charts}
%   \begin{figure}
%     \begin{tikzpicture}
%       \begin{axis}[
%         mbarplot,
%         xlabel={Foo},
%         ylabel={Bar},
%         width=0.9\textwidth,
%         height=6cm,
%       ]

%       \addplot plot coordinates {(1, 20) (2, 25) (3, 22.4) (4, 12.4)};
%       \addplot plot coordinates {(1, 18) (2, 24) (3, 23.5) (4, 13.2)};
%       \addplot plot coordinates {(1, 10) (2, 19) (3, 25) (4, 15.2)};

%       \legend{lorem, ipsum, dolor}

%       \end{axis}
%     \end{tikzpicture}
%   \end{figure}
% \end{frame}

% \begin{frame}[fragile]{Code Snippets}
%     Code snippets with syntax highlighting can be embedded using then \verb|minted| package:
%     \begin{minted}[escapeinside=||]{haskell}
% newtype Lasagne = Lasagne Int
%     deriving (Show, Num)

% -- The stacking operating can be considered integer 
% -- addition of the number of layers
% instance Semigroup Lasagne where
%     (<>) = (+)
%     \end{minted}
% \end{frame}


\end{document}
